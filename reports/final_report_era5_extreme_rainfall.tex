\documentclass[conference]{IEEEtran}
\IEEEoverridecommandlockouts

\usepackage{cite}
\usepackage{amsmath,amssymb,amsfonts}
\usepackage{algorithmic}
\usepackage{graphicx}
\usepackage{textcomp}
\usepackage{xcolor}
\usepackage{url}

\def\BibTeX{{\rm B\kern-.05em{\sc i\kern-.025em b}\kern-.08em
    T\kern-.1667em\lower.7ex\hbox{E}\kern-.125emX}}

\begin{document}

\title{Machine Learning-Based Nowcasting of Extreme Rainfall over Aceh from ERA5 Reanalysis\\[0.4em]
\large Application Project -- Physical Sciences / Machine Learning}

\author{
\IEEEauthorblockN{Muhammad Raza Adzani}
\IEEEauthorblockA{
\textit{Department of Informatics} \\
\textit{Syiah Kuala University}\\
Banda Aceh, Indonesia \\
email: Raza.a22@mhs.usk.ac.id
}
\and
\IEEEauthorblockN{Ahmad Siddiq}
\IEEEauthorblockA{
\textit{Department of Informatics} \\
\textit{Syiah Kuala University} \\
Banda Aceh, Indonesia \\
email: m.siddiq2@mhs.usk.ac.id
}

}

\maketitle

\begin{abstract}
Short-term predictions of extreme rainfall are crucial for flood early warning in coastal regions such as Aceh, Indonesia, yet operational systems often rely on numerical weather prediction models with limited local calibration. In this work, we formulate 3-hour extreme rainfall nowcasting over a small domain around Banda Aceh as a supervised binary classification problem on a 0.25$^\circ$ $\times$ 0.25$^\circ$ latitude--longitude grid. Using ERA5 single-level reanalysis from 2020 to 2024, we construct a dataset of 365{,}400 samples combining total precipitation, runoff, near-surface wind, temperature, soil moisture and simple time-series features including short-term rainfall lags and rolling means. Extreme events are defined as 3-hour-ahead accumulated precipitation above the empirical 95th percentile over the study period. We compare logistic regression and random forest classifiers and adopt a chronological split by year to emulate an operational setting with a held-out test year. On the validation set, the random forest model achieves a ROC AUC of about 0.90 and clearly outperforms logistic regression in terms of extreme-event recall. After tuning the decision threshold to prioritize the rare extreme class, the random forest attains an overall accuracy of approximately 0.93 on the test year, with a recall of about 0.58 and precision of about 0.33 for extreme events. Finally, we implement a simple web-based prototype using Streamlit that visualizes grid-scale risk maps and allows manual input of meteorological conditions for demonstration and educational purposes.
\end{abstract}

\begin{IEEEkeywords}
extreme rainfall, nowcasting, random forest, ERA5 reanalysis, Aceh, Indonesia, machine learning
\end{IEEEkeywords}

\section{Introduction}

Extreme rainfall is one of the main triggers of floods and landslides across Indonesia, and the frequency of hydrometeorological disasters has increased in recent years. Northern Sumatra, including Aceh Province at the northern tip of the island, is particularly exposed because intense monsoon rainfall, complex topography and rapid land-use change combine to produce frequent flash floods and landslides. Recent events in North Sumatra and Aceh, where short-lived but intense storms caused fatalities and displaced thousands of people, illustrate how quickly extreme rainfall can escalate into a large-scale humanitarian crisis.

Operational early-warning systems in Indonesia still face several challenges in anticipating such short-lived but high-impact events. Gauge networks are sparse in many districts, radar coverage is limited, and numerical weather prediction (NWP) guidance can be biased or under-dispersive when forecasting local extremes at very short lead times. As a result, local decision makers often have to choose between issuing conservative warnings based on qualitative judgment, or waiting for more certainty at the cost of valuable lead time. Reanalysis products such as ERA5, which provide physically consistent atmospheric fields at hourly resolution on a 0.25$^\circ$ grid globally, offer an alternative data source that can be exploited using data-driven methods to derive local indicators of extreme rainfall risk.

In parallel, there is a growing body of work on machine-learning approaches for precipitation forecasting and post-processing of NWP or reanalysis output. Random forest and related ensemble tree methods have been used to improve exceedance forecasts for heavy precipitation and quantitative precipitation estimation, often showing higher skill than logistic regression and raw ensemble guidance in distinguishing extreme from non-extreme events~\cite{b1,b8}. Other studies correct biases in ERA5 precipitation using machine learning and demonstrate substantial improvements in hydrological simulations, highlighting the value of reanalysis-driven ML approaches for rainfall-related applications~\cite{b2}. Logistic regression, random forests and gradient-boosting methods have also been applied to rainfall occurrence or intensity classes at daily to monthly scales in both tropical and temperate locations, demonstrating that relatively simple models can capture non-linear relationships between large-scale predictors and local rainfall~\cite{b3,b4,b10}.

Beyond tree-based methods, deep learning architectures such as convolutional neural networks, sequence models and deep generative models have been explored for precipitation nowcasting using radar and satellite input, and several studies report improved skill for short lead times and for intense events compared with classical baselines~\cite{b5,b6,b7,b9}. These approaches include convolutional neural networks trained on radar composites, convolutional long short-term memory (ConvLSTM) networks that model the spatiotemporal evolution of precipitation fields, and generative models that produce probabilistic nowcasts.

For the Indonesian region, several machine-learning studies have targeted rainfall estimation and flood risk but typically without explicit short-lead extreme rainfall classification on reanalysis grids. Existing work often focuses on daily rainfall prediction at station scale or on river-basin flood/no-flood outcomes using aggregated rainfall indicators and hydrological features, rather than grid-scale nowcasting of three-hour extreme precipitation. This motivates the development of a locally tailored, reanalysis-driven nowcasting framework that focuses specifically on extreme events over Aceh.

This paper proposes a machine-learning framework for grid-based early warning of extreme rainfall three hours ahead over Aceh Province, using only variables available from the ERA5 single-level reanalysis. We define extreme rainfall at each grid cell as the upper 5th percentile of accumulated total precipitation three hours ahead, computed over a multi-year baseline, and cast the problem as a binary classification task. Our approach concentrates on a small $5 \times 5$ grid (0.25$^\circ$ spacing) covering coastal and inland parts of Aceh, and uses a combination of instantaneous meteorological variables, short-term lags of total precipitation, and simple temporal encodings (hour of day, day of year) as predictors. A random forest classifier is trained and evaluated using several years of hourly ERA5 data, and the decision threshold is tuned to favor high recall for the extreme class, reflecting the preference of local stakeholders to avoid missed extreme events even at the cost of more false alarms.

The main contributions of this work are threefold. First, we deploy the trained random-forest model in a lightweight web application built with Streamlit, which allows users in Aceh to explore grid-scale maps of predicted extreme-rainfall probability and to input hypothetical meteorological conditions for specific grid points. Second, we develop a reproducible pipeline that converts raw ERA5 hourly single-level fields into a labelled dataset for three-hour-ahead extreme-rainfall classification at grid scale, including feature engineering and class-imbalance handling tailored to a small tropical domain. Third, we show that a relatively simple random forest model, trained purely on reanalysis variables, can achieve an area under the ROC curve (AUC) of about 0.90 on an independent test set while maintaining recall above 0.55 for the extreme class at a practically chosen probability threshold.

\section{Related Work}

Machine learning has been increasingly adopted to improve short-range extreme precipitation forecasts beyond traditional statistical and NWP-based methods. Random forest classifiers have been used to predict the probability that 24-hour precipitation exceeds extreme thresholds, showing clear gains over logistic regression and raw ensemble guidance, especially for high-impact events~\cite{b1}. Random forests and related ensembles have also been applied to quantitative precipitation estimation from multi-source radar networks in complex terrain, where they outperform conventional QPE methods and provide better representation of intense rainfall~\cite{b8}. These results highlight the suitability of tree-based models for handling non-linear interactions among multiple predictors in heavy-rainfall settings.

On longer time scales, Yang et al.\ proposed a method for monthly extreme precipitation forecasting with physically interpretable predictors and tree-based models, reporting improved skill for several extreme precipitation indices over China~\cite{b3}. Sun et al.\ showed that machine-learning correction of ERA5 precipitation substantially improves hydrological flow simulations over high-mountain Asian basins, emphasizing the value of post-processing reanalysis precipitation for downstream water-resources applications~\cite{b2}. Chkeir et al.\ investigated nowcasting of extreme rain and extreme wind speed using several machine-learning techniques and different input datasets, including radar and NWP variables, and showed that data-driven models can provide useful early warnings for high-impact weather~\cite{b4}. More recently, machine learning has also been used for smart downscaling of daily precipitation extremes, linking coarse-scale predictors to local extreme indices~\cite{b10}.

Beyond tree-based methods, deep learning architectures have been widely explored for precipitation nowcasting and extreme rainfall classification. Convolutional LSTM networks were introduced as a spatiotemporal model for radar-based precipitation nowcasting and demonstrated improved performance over optical-flow-based extrapolation, particularly for convective events~\cite{b9}. RainNet is a fully convolutional neural network that performs radar-based precipitation nowcasting on gridded fields and achieves competitive skill at short lead times~\cite{b6}. More recently, generative models trained on large archives of radar data have been shown to deliver skillful probabilistic nowcasts of precipitation intensity at high spatial and temporal resolution~\cite{b5}. Other works combine convolutional neural networks and recurrent architectures for immediate precipitation forecasting, showing that hybrid CNN--LSTM models can exploit both spatial patterns and temporal evolution of rainfall fields~\cite{b7}.

For the Indonesian region and other tropical areas, many machine-learning studies have focused on rainfall prediction and flood risk at daily or coarser time scales, often using station data, satellite estimates or basin-aggregated indicators combined with decision trees, random forests and neural networks. These studies generally aim at predicting rainfall categories or flood/no-flood outcomes rather than explicitly targeting grid-scale, three-hour-ahead extreme rainfall classification on reanalysis fields. Compared with these works, our study focuses on a data-scarce but high-impact region---Aceh Province on the northern tip of Sumatra---and formulates a three-hour-ahead classification of extreme precipitation (top 5\% of ERA5 total precipitation) on a 0.25$^\circ$ grid using only single-level reanalysis variables. We use a relatively simple but interpretable random-forest classifier, explicitly tune the decision threshold to favour high recall of extremes for early-warning purposes in Aceh, and implement a lightweight web prototype that allows local stakeholders to explore grid-scale risk maps and perform manual ``what-if'' scenarios at locations of interest.

\section{Study Area and Data}

\subsection{Study Area}

The study domain covers a small region around Banda Aceh in the northern part of Sumatra, Indonesia. In latitude--longitude space, we consider the range 5.0$^\circ$N to 6.0$^\circ$N and 95.0$^\circ$E to 96.0$^\circ$E, which corresponds to a $5 \times 5$ grid at 0.25$^\circ$ spacing. This domain includes coastal and inland areas that are exposed to intense monsoon rainfall and flash flooding. The spatial resolution of 0.25$^\circ$ is consistent with the native horizontal resolution of the ERA5 reanalysis used in this work.

\subsection{ERA5 Single-Level Data}

We use hourly single-level fields from the ERA5 reanalysis produced by the European Centre for Medium-Range Weather Forecasts (ECMWF). Data are downloaded from the Copernicus Climate Data Store as yearly NetCDF4 files for the period 2020--2024 and then merged using Python. From the full ERA5 catalogue, we extract the following variables on the specified grid:

\begin{itemize}
    \item \texttt{tp}: total precipitation (m) accumulated over the previous hour,
    \item \texttt{ro}: runoff (m),
    \item \texttt{u10}, \texttt{v10}: 10 m wind components (m/s),
    \item \texttt{t2m}: 2 m air temperature (K),
    \item \texttt{swvl1}: volumetric soil water in the top layer,
    \item \texttt{valid\_time}: timestamp of the field.
\end{itemize}

Each record is associated with a grid point defined by latitude, longitude and time, and includes internal ERA5 indices such as \texttt{number} and \texttt{expver}. After merging the hourly files and subsetting to 3-hourly intervals, the resulting dataset contains 365{,}400 samples (5 years $\times$ 4 time steps per day $\times$ 365 or 366 days $\times$ 25 grid cells), with no missing values and a full $5 \times 5$ grid at each time step.

\subsection{Derived Variables and Label Definition}

From the basic variables we derive several additional predictors and the target label. Wind speed is computed from the 10 m wind components as
\begin{equation}
    \text{wind\_speed} = \sqrt{\text{u10}^2 + \text{v10}^2}.
\end{equation}
We also extract temporal features from the timestamp: year, month, day, hour, day of week and day of year.

To formulate a three-hour-ahead prediction problem, we define \texttt{tp\_next} at each grid point and time as the total precipitation accumulated over the next 3 hours (obtained by shifting and summing the hourly \texttt{tp} series). Extreme events are defined using the empirical 95th percentile of \texttt{tp\_next} over all years and grid cells. The binary target label \texttt{is\_extreme\_next} is set to 1 if \texttt{tp\_next} at that grid point and time is greater than or equal to this percentile, and 0 otherwise. This yields an extreme-event frequency of about 5\% in the full dataset.

A summary of the resulting dataset is provided in Table~\ref{tab:dataset}.

\begin{table}[htbp]
\caption{Summary of the ERA5-based dataset for Aceh}
\label{tab:dataset}
\centering
\begin{tabular}{lc}
\hline
Item & Value \\
\hline
Period & 2020--2024 \\
Domain & 5.0--6.0$^\circ$N, 95.0--96.0$^\circ$E \\
Grid resolution & 0.25$^\circ \times$ 0.25$^\circ$ (5 $\times$ 5 cells) \\
Temporal resolution & 3-hourly \\
Total samples & 365{,}400 \\
Extreme-event definition & top 5\% of 3 h tp\_next \\
Extreme fraction (all years) & $\approx$ 5\% \\
Train samples (2020--2022) & 219{,}150 \\
Validation samples (2023) & 73{,}000 \\
Test samples (2024) & 73{,}175 \\
\hline
\end{tabular}
\end{table}

\section{Methodology}

\subsection{Problem Formulation}

We cast the three-hour-ahead extreme rainfall prediction task as a binary classification problem at the grid-cell level. For each grid point and time $t$, we aim to predict whether the three-hour-ahead precipitation accumulation \texttt{tp\_next} will exceed the extreme threshold. The classifier outputs a probability $p$ that the event is extreme; a probability threshold $\tau$ is then applied to convert $p$ into a binary decision (extreme vs.\ non-extreme).

\subsection{Feature Engineering}

To provide the classifier with information about recent meteorological conditions and temporal cycles, we construct a feature vector with 16 predictors:
\begin{itemize}
    \item spatial coordinates: latitude, longitude;
    \item current meteorology: \texttt{tp}, \texttt{ro}, \texttt{u10}, \texttt{v10}, \texttt{t2m}, \texttt{swvl1}, \texttt{wind\_speed};
    \item temporal encodings: hour\_sin, hour\_cos, doy\_sin, doy\_cos;
    \item short-term precipitation history: \texttt{tp\_lag1}, \texttt{tp\_lag2}, \texttt{tp\_roll3\_mean}.
\end{itemize}

The lagged features are computed for each grid point by sorting records in time and taking the value of \texttt{tp} one and two time steps (3 and 6 hours) before $t$. The rolling mean \texttt{tp\_roll3\_mean} is the average of the current \texttt{tp} and the two previous time steps. Records at the beginning of each grid-point time series that do not have sufficient history are dropped. Diurnal and seasonal cycles are encoded using sine and cosine transforms of the hour of day and the day of year:
\begin{equation}
\text{hour\_sin} = \sin\left(2\pi \frac{\text{hour}}{24}\right), \quad
\text{hour\_cos} = \cos\left(2\pi \frac{\text{hour}}{24}\right),
\end{equation}
\begin{equation}
\text{doy\_sin} = \sin\left(2\pi \frac{\text{doy}}{366}\right), \quad
\text{doy\_cos} = \cos\left(2\pi \frac{\text{doy}}{366}\right).
\end{equation}

\subsection{Train--Validation--Test Split}

To emulate an operational setting and avoid temporal leakage, we split the data chronologically by year after computing labels and features. Samples from 2020--2022 are used for training, 2023 for validation (model selection and threshold tuning), and 2024 is held out as an independent test set. This yields 219{,}150 training samples, 73{,}000 validation samples and 73{,}175 test samples, with class proportions for the extreme label close to 5\% in each split.

\subsection{Models and Training}

We consider two classifiers implemented with scikit-learn: logistic regression (LR) as a linear baseline and random forest (RF) as the main non-linear model. Both models are trained to predict the probability of the extreme class. We do not perform extensive hyperparameter tuning in this study; instead, we use standard settings for LR with regularization and a reasonably large number of trees for RF. Class imbalance is implicitly handled by focusing evaluation on metrics for the extreme class and by later tuning the decision threshold.

\subsection{Evaluation Metrics}

Given the strong class imbalance, we emphasize metrics that capture performance on the rare extreme events. For each model and data split, we compute:
\begin{itemize}
    \item overall accuracy,
    \item precision, recall and F$_1$-score for the extreme class,
    \item area under the receiver operating characteristic (ROC) curve (AUC).
\end{itemize}

The ROC curve and AUC summarize the trade-off between true-positive and false-positive rates across thresholds. However, in practice a single threshold must be chosen. We therefore perform a threshold sweep on the validation set, varying $\tau$ from 0.05 to 0.95 in steps of 0.05 and computing precision, recall and F$_1$-score for the extreme class at each value. This analysis guides the choice of the operational threshold.

\section{Results and Discussion}

\subsection{Overall Classification Performance}

We first compare logistic regression (LR) and random forest (RF) as baseline and main classifiers. Using the default probability threshold $\tau = 0.5$ on the validation set, the LR model achieves high overall accuracy but fails to reliably detect rare extreme events: the recall for the extreme class remains low despite a reasonably high ROC AUC. In contrast, the RF model consistently yields higher recall and F$_1$-score for extreme rainfall while also attaining a validation ROC AUC of about 0.90.

Because only around 5\% of samples correspond to extreme events, we place greater emphasis on the performance of the positive class. Table~\ref{tab:performance} summarizes key metrics for selected model--threshold combinations. The RF model clearly outperforms LR in terms of F$_1$-score for the extreme class on the validation set, while maintaining comparable or better overall accuracy.

\begin{table}[htbp]
\caption{Summary of model performance for the extreme class}
\label{tab:performance}
\centering
\begin{tabular}{lccccc}
\hline
Model / set & Acc. & Prec. & Rec. & F$_1$ & AUC \\
\hline
LR (val, $\tau{=}0.5$) 
  & 0.856 & 0.169 & 0.686 & 0.271 & 0.845 \\
RF (val, $\tau{=}0.15$) 
  & 0.922 & 0.266 & 0.574 & 0.364 & 0.899 \\
RF (test, $\tau{=}0.15$) 
  & 0.926 & 0.332 & 0.581 & 0.422 & --- \\
\hline
\end{tabular}
\end{table}

On the held-out test year, the RF model with the selected threshold achieves an overall accuracy of approximately 0.93. For the extreme class, it attains a precision of about 0.33, a recall of about 0.58 and an F$_1$-score of about 0.42. These results indicate that the model can correctly identify more than half of the extreme three-hour rainfall events while still filtering out a substantial fraction of non-extreme cases, which is a desirable property for early-warning applications.

Fig.~\ref{fig:roc} shows the ROC curve of the RF model on the validation set, confirming good discriminative ability with an AUC close to 0.90.

\begin{figure}[htbp]
\centerline{\includegraphics[width=\linewidth]{images/roc_curve_val.png}}
\caption{ROC curve of the random forest model on the validation set.}
\label{fig:roc}
\end{figure}

\subsection{Threshold Analysis}

As the RF model produces class probabilities, the choice of decision threshold $\tau$ has a direct impact on the trade-off between missed extreme events and false alarms. Fig.~\ref{fig:threshold_curve} summarizes the results of a threshold sweep on the validation set, where $\tau$ is varied from 0.05 to 0.95 and precision, recall and F$_1$-score for the extreme class are computed at each value.

For very low thresholds ($\tau \approx 0.05$), the model triggers many alarms, achieving very high recall (above 0.8) but very low precision (around 0.15). As the threshold increases towards $\tau = 0.2$, recall decreases while precision improves, and the F$_1$-score reaches its maximum in the range $\tau = 0.15$--0.2. For thresholds larger than about 0.4, recall collapses as the model becomes overly conservative and most extreme events are no longer flagged.

In an early-warning context, missing a truly extreme event is typically more critical than issuing a few unnecessary alerts. We therefore select $\tau = 0.15$ as a compromise that favors detecting extreme rainfall while maintaining a reasonable false-alarm rate. This choice lies near the F$_1$ peak and provides a useful balance between sensitivity and specificity for the extreme class.

\begin{figure}[htbp]
\centerline{\includegraphics[width=\linewidth]{images/rf_threshold_tuning_val.png}}
\caption{Precision, recall and F$_1$-score for the extreme class on the validation set as a function of the probability threshold $\tau$ for the random forest model.}
\label{fig:threshold_curve}
\end{figure}

\subsection{Confusion Matrix Analysis}

To better understand the types of errors made by the selected model, we inspect confusion matrices for the RF with $\tau = 0.15$ on the validation and test sets. On the validation set, the model reduces the number of missed extreme events compared to using the default threshold of 0.5, but at the cost of additional false alarms. On the held-out test year, the confusion matrix in Fig.~\ref{fig:cm_test} shows that the model correctly classifies the vast majority of non-extreme cases (true negatives) and detects more than half of the extreme events (true positives).

The non-negligible number of false positives is consistent with the moderate precision value and reflects a cautious behavior: the system is willing to raise an alarm in some borderline situations in order to avoid missing too many truly extreme three-hour rainfall events. From an operational perspective, this behavior is often acceptable or even desirable for early warning, where failing to warn about a genuinely extreme event can be more harmful than issuing several alarms that later turn out to be non-extreme.

\begin{figure}[htbp]
\centerline{\includegraphics[width=\linewidth]{images/confusion_matrix_rf_test_thr_0.15.png}}
\caption{Confusion matrix of the random forest model with $\tau = 0.15$ on the test set.}
\label{fig:cm_test}
\end{figure}

\subsection{Feature Importance Analysis}

To gain insight into which predictors drive the RF decisions, we examine the mean decrease in impurity feature importance scores provided by the trained model. Fig.~\ref{fig:feature_importance} shows the top 15 features ranked by their importance for predicting three-hour-ahead extreme rainfall.

The most influential predictor is the current total precipitation \texttt{tp}, followed by the three-step rolling mean \texttt{tp\_roll3\_mean} and the first lag \texttt{tp\_lag1}. This pattern indicates that both the instantaneous intensity and the recent accumulation of rainfall strongly condition the likelihood of an extreme event in the next three hours, which is physically consistent with the tendency of convective systems to persist and organize over several hours.

Among the non-precipitation variables, top-layer soil moisture (\texttt{swvl1}) and near-surface air temperature (\texttt{t2m}) show substantial importance. Higher soil moisture may reflect antecedent wet conditions and a saturated land surface, which increase the impact and likelihood of runoff when heavy rainfall occurs. Temperature can modulate atmospheric stability and moisture capacity, and thus indirectly influence convective activity.

Dynamical features such as 10 m wind speed and direction (\texttt{wind\_speed}, \texttt{u10}, \texttt{v10}) also contribute meaningfully, together with the cyclical encodings of hour-of-day and day-of-year (\texttt{hour\_sin}, \texttt{hour\_cos}, \texttt{doy\_sin}, \texttt{doy\_cos}). This suggests that the model exploits both the prevailing low-level flow and the typical diurnal and seasonal cycles of rainfall in the maritime tropics. In contrast, purely spatial information such as latitude has relatively low importance, which is expected given the small size of the study domain.

\begin{figure}[htbp]
\centerline{\includegraphics[width=\linewidth]{images/rf_feature_importance_top15.png}}
\caption{Top 15 feature importance scores for the random forest model.}
\label{fig:feature_importance}
\end{figure}

\subsection{Spatial Patterns and Web Prototype}

Beyond aggregate skill scores, it is important to inspect how the model behaves in space and time. For a given valid time, we apply the trained RF to all grid cells and visualize the predicted probability of extreme rainfall on the $5 \times 5$ ERA5 grid over Aceh. The resulting probability fields show that the model can assign elevated risk to clusters of neighboring cells, for example along the coast, while keeping inland or offshore cells at lower risk at the same time. Such patterns are consistent with localized convective systems affecting only part of the domain and demonstrate that the model does not simply predict a uniform risk over all grid points.

To make these results more accessible to non-expert users, we implement a simple web-based prototype using the Streamlit framework. The application provides two main views: (i) a ``historical map'' tab, which displays the model-predicted probability and binary extreme/non-extreme classification for each grid cell at a selected historical time, together with a table of feature values; and (ii) a ``manual input'' tab, where users can specify a grid location, time and meteorological conditions (including recent rainfall history) and obtain a predicted probability and binary decision. Although this prototype is not intended as an operational warning system, it demonstrates how the machine-learning model can be integrated into an interactive interface for exploration and communication of extreme rainfall risk in Aceh.

\section{Conclusion and Future Work}

In this paper, we developed a machine-learning framework for three-hour-ahead nowcasting of extreme rainfall over Aceh Province using ERA5 single-level reanalysis. Starting from hourly ERA5 fields downloaded as yearly NetCDF4 files from the Copernicus Climate Data Store, we constructed a five-year dataset on a 0.25$^\circ \times 0.25^\circ$ grid covering a small coastal and inland domain around Banda Aceh. The raw data were transformed into a labelled binary classification problem by defining extreme events as grid-point total precipitation in the next three hours exceeding the empirical 95th percentile, and by engineering a set of physically motivated predictors including precipitation history, soil moisture, near-surface wind and temperature, and cyclical encodings of the diurnal and seasonal cycles.

Among the models considered, a RF classifier trained on these features achieved the best balance between overall skill and sensitivity to rare extremes. On an independent test year, the selected model reached an area under the ROC curve of about 0.90 and an overall accuracy of approximately 0.93. When operated at a probability threshold of $\tau = 0.15$, chosen based on a validation-set threshold sweep, the model maintained a recall of about 0.58 and a precision of about 0.33 for the extreme class, correctly detecting more than half of the extreme three-hour rainfall events while keeping false alarms at a manageable level. Feature-importance analysis highlighted the dominant role of current and recent precipitation, complemented by soil moisture, near-surface temperature, wind and time-of-day / time-of-year information. Spatial maps of predicted probability showed coherent patterns across the $5 \times 5$ grid and illustrated how the model differentiates between coastal and inland grid cells at specific times.

A key practical contribution of this work is the integration of the trained RF into a lightweight web prototype built with Streamlit. The application allows users to visualize grid-scale maps of predicted extreme-rainfall probability for selected historical times and to manually input meteorological conditions at specific grid cells to obtain three-hour-ahead risk estimates. This demonstrates how a reanalysis-driven machine-learning model can be transformed into an interactive tool that supports local awareness and discussion of extreme rainfall risk in Aceh, even if it is not yet an operational warning system.

Several limitations of the current study suggest directions for future work. First, we rely solely on ERA5 single-level variables at 0.25$^\circ$ resolution without assimilating local rain-gauge, radar or hydrological information; incorporating such observations would likely improve calibration and local relevance. Second, we focus on a single RF model and a fixed P95 threshold for defining extremes. Exploring alternative algorithms (e.g., gradient-boosted trees, distributional forests or deep learning architectures) and more impact-oriented thresholds linked to flood reports could provide additional insight. Third, our framework operates independently at each grid cell and does not explicitly enforce spatial consistency or quantify predictive uncertainty. Extending the model to account for spatial dependence and to output calibrated probabilistic forecasts would be valuable for risk communication. Finally, closer collaboration with local agencies in Aceh, such as disaster-management and meteorological services, is needed to co-design decision thresholds, visualization layouts and operating procedures so that future versions of the system can better support real-world early-warning and response.

\section*{Acknowledgment}

The authors thank the European Centre for Medium-Range Weather Forecasts (ECMWF) and the Copernicus Climate Data Store for providing access to ERA5 reanalysis data.

\begin{thebibliography}{00}

\bibitem{b1}
G.~R. Herman and R.~S. Schumacher, ``Money {D}oesn't {G}row on {T}rees, but {F}orecasts {D}o: {F}orecasting {E}xtreme {P}recipitation with {R}andom {F}orests,'' \emph{Monthly Weather Review}, vol.~146, no.~5, pp.~1571--1600, 2018, doi: 10.1175/MWR-D-17-0250.1.

\bibitem{b2}
H.~Sun, T.~Yao, F.~Su, Z.~He, G.~Tang, N.~Li, B.~Zheng, J.~Huang, F.~Meng, T.~Ou, and D.~Chen, ``Corrected {ERA5} precipitation by machine learning significantly improved flow simulations for the {T}hird {P}ole basins,'' \emph{Journal of Hydrometeorology}, vol.~23, no.~10, pp.~1663--1679, 2022, doi: 10.1175/JHM-D-22-0015.1.

\bibitem{b3}
B.~Yang, L.~Chen, V.~P. Singh, B.~Yi, Z.~Leng, J.~Zheng, and Q.~Song, ``A method for monthly extreme precipitation forecasting with physical explanations,'' \emph{Water}, vol.~15, no.~8, art.~1545, 2023, doi: 10.3390/w15081545.

\bibitem{b4}
S.~Chkeir, A.~Anesiadou, A.~Mascitelli, and R.~Biondi, ``Nowcasting extreme rain and extreme wind speed with machine learning techniques applied to different input datasets,'' \emph{Atmospheric Research}, vol.~282, art.~106548, 2023, doi: 10.1016/j.atmosres.2022.106548.

\bibitem{b5}
S.~Ravuri \emph{et al.}, ``Skillful precipitation nowcasting using deep generative models of radar,'' \emph{Nature}, vol.~597, pp.~672--677, 2021, doi: 10.1038/s41586-021-03854-z.

\bibitem{b6}
G.~Ayzel, M.~Heistermann, and S.~Sorokin, ``RainNet v1.0: A convolutional neural network for radar-based precipitation nowcasting,'' \emph{Geoscientific Model Development}, vol.~13, no.~6, pp.~2631--2644, 2020, doi: 10.5194/gmd-13-2631-2020.

\bibitem{b7}
F.~J. Gamboa-Villafruela, ``Immediate precipitation forecasting using deep learning: A convolutional neural network and long short-term memory approach,'' \emph{Proceedings}, vol.~8, no.~1, art.~33, 2021, doi: 10.3390/proceedings8010033.

\bibitem{b8}
D.~Wolfensberger and A.~Feinberg, ``RainForest: A random forest algorithm for quantitative precipitation estimation over Switzerland,'' \emph{Atmospheric Measurement Techniques}, vol.~14, pp.~3169--3193, 2021, doi: 10.5194/amt-14-3169-2021.

\bibitem{b9}
X.~Shi, Z.~Chen, H.~Wang, D.-Y. Yeung, W.-K. Wong, and W.-C. Woo, ``Convolutional LSTM network: A machine learning approach for precipitation nowcasting,'' in \emph{Advances in Neural Information Processing Systems}, vol.~28, 2015.

\bibitem{b10}
X.~Shi, ``Enabling smart dynamical downscaling of extreme precipitation events with machine learning,'' \emph{Geophysical Research Letters}, vol.~47, no.~19, e2020GL090309, 2020, doi: 10.1029/2020GL090309.

\end{thebibliography}

\end{document}
