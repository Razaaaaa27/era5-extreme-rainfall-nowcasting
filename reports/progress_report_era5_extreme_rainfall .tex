\documentclass[conference]{IEEEtran}
\IEEEoverridecommandlockouts

\usepackage{cite}
\usepackage{amsmath,amssymb,amsfonts}
\usepackage{algorithmic}
\usepackage{graphicx}
\usepackage{textcomp}
\usepackage{xcolor}
\usepackage{url}

\def\BibTeX{{\rm B\kern-.05em{\sc i\kern-.025em b}\kern-.08em
    T\kern-.1667em\lower.7ex\hbox{E}\kern-.125emX}}

\begin{document}

\title{Progress Report: Machine Learning--Based Nowcasting of Extreme Rainfall over Aceh Using ERA5 Reanalysis}

\author{
\IEEEauthorblockN{Muhammad Raza Adzani}
\IEEEauthorblockA{
\textit{Department of Informatics} \\
\textit{Syiah Kuala University} \\
Banda Aceh, Indonesia \\
email: raza.a22@mhs.usk.ac.id
}
\and
\IEEEauthorblockN{Ahmad Siddiq}
\IEEEauthorblockA{
\textit{Department of Informatics} \\
\textit{Syiah Kuala University} \\
Banda Aceh, Indonesia \\
email: m.siddiq2@mhs.usk.ac.id
}

}

\maketitle

\begin{abstract}
Short-term prediction of extreme rainfall is essential for flood early warning in regions such as Aceh, Indonesia, where intense convective storms frequently cause flash floods. This progress report presents the current status of a machine learning--based framework for three-hour-ahead extreme rainfall nowcasting using ERA5 single-level reanalysis data. The problem is formulated as a grid-based binary classification task on a 0.25$^\circ$ $\times$ 0.25$^\circ$ domain. At this stage, data preprocessing, feature engineering with SMOTE for class imbalance handling, and advanced model implementation have been completed. Three machine learning approaches have been evaluated: XGBoost as an advanced gradient boosting classifier, a multi-layer perceptron (MLP) as an LSTM-like model, and an ensemble combining both. Preliminary results using three-fold cross-validation show that XGBoost achieves exceptional performance with accuracy exceeding 0.95 and ROC AUC above 0.99 for extreme event detection. Ongoing work focuses on final evaluation, model interpretation, and developing a web-based prototype using Streamlit to support early-warning applications.
\end{abstract}

\begin{IEEEkeywords}
extreme rainfall, nowcasting, machine learning, XGBoost, LSTM, ensemble learning, ERA5 reanalysis, progress report
\end{IEEEkeywords}

\section{Introduction}

Extreme rainfall remains one of the main drivers of floods and landslides in Indonesia, with Aceh Province being particularly vulnerable due to intense monsoonal precipitation and limited local monitoring infrastructure. Short-lived but high-intensity rainfall events often escalate rapidly into disasters, highlighting the need for timely early-warning information.

Existing early-warning systems rely primarily on numerical weather prediction models and sparse observational networks, which may show limited skill at very short lead times and fine spatial scales. As discussed in the proposal stage of this project, reanalysis datasets such as ERA5 offer an opportunity to explore data-driven approaches that learn empirical relationships between atmospheric conditions and subsequent rainfall extremes. This progress report summarizes the work completed to date toward developing a machine learning--based nowcasting framework for extreme rainfall over Aceh.

\section{Related Work}

Machine learning approaches for precipitation forecasting and extreme rainfall prediction have been widely explored in recent years. Random forest classifiers have demonstrated improved performance over logistic regression for predicting exceedance of heavy precipitation thresholds and for distinguishing extreme from non-extreme events~\cite{b1,b8}. Other studies have applied machine learning to correct biases in reanalysis precipitation, leading to improved hydrological simulations~\cite{b2}.

Beyond tree-based methods, deep learning architectures such as convolutional and recurrent neural networks have been used for precipitation nowcasting with radar and satellite data~\cite{b5,b6,b7,b9}. While these approaches achieve high skill, they typically require dense observational coverage. For data-scarce regions such as Aceh, relatively simple but robust models driven by reanalysis variables remain an attractive alternative~\cite{b3,b4,b10}. This project builds on these ideas by focusing on short-lead extreme rainfall classification using ERA5 data.

\section{Methodology}

\subsection{Data and Study Area}

ERA5 single-level reanalysis data from 2020 to 2024 are used over a small domain surrounding Banda Aceh (5.0$^\circ$--6.0$^\circ$N, 95.0$^\circ$--96.0$^\circ$E). The spatial resolution of 0.25$^\circ$ yields a $5 \times 5$ grid covering coastal and inland areas. Hourly fields are aggregated to three-hourly intervals, resulting in a dataset of 365{,}400 grid-time samples.

\subsection{Problem Formulation and Features}

The task is formulated as a binary classification problem, where the goal is to predict whether three-hour-ahead accumulated precipitation exceeds the empirical 95th percentile. Features implemented at this stage include near-surface meteorological variables, lagged precipitation values, rolling means, and cyclical encodings of diurnal and seasonal time scales.

\subsection{Models and Data Splitting}

Three advanced machine learning models have been implemented: (1) XGBoost, an extreme gradient boosting classifier with 300 estimators and early stopping, optimized for handling non-linear relationships; (2) a multi-layer perceptron with hidden layers (64, 32, 16 neurons) serving as an LSTM-like deep learning model; and (3) an ensemble model that combines predictions from both XGBoost and MLP through soft voting. To address the severe class imbalance ($\sim$5\% extreme events), SMOTE (Synthetic Minority Over-sampling Technique) is applied to the training data. A three-fold cross-validation strategy with chronological splits is used to ensure robust evaluation and temporal independence.

\section{Preliminary Experiments}

Preliminary experiments across three folds have been conducted to assess model performance. XGBoost consistently achieves exceptional results, with accuracy ranging from 0.945 to 0.975, precision from 0.863 to 0.924, recall from 0.900 to 0.961, and ROC AUC from 0.989 to 0.996 across the three folds. The LSTM-like MLP model shows competitive but lower performance, with accuracy between 0.890 and 0.951, while the ensemble provides a balanced middle ground.

These preliminary results demonstrate that advanced tree-based methods combined with SMOTE for class balancing can effectively identify extreme rainfall events without requiring manual threshold tuning. XGBoost's ability to correctly detect over 96\% of extreme events (on the best fold) while maintaining high precision (92.4\%) is particularly promising for early-warning applications, where both sensitivity and reliability are crucial. The consistent high performance across all folds indicates good generalization capability.

\section{Next Steps}

Several tasks remain to be completed in the final phase of the project. First, comprehensive analysis of feature importance will be conducted to understand which meteorological variables drive XGBoost's predictions. Second, detailed confusion matrix analysis and error characterization will be performed to identify potential failure modes. Third, a web-based prototype using Streamlit will be developed, providing interactive grid-scale risk maps and manual input capabilities for demonstration and educational purposes. Fourth, spatial pattern analysis will be conducted to assess how the model differentiates risk across coastal and inland grid cells. Finally, the complete pipeline, results, and web application will be documented in the final report with recommendations for operational deployment and future enhancements such as incorporating local observations and exploring spatiotemporal deep learning architectures.

\begin{thebibliography}{00}

\bibitem{b1}
G.~R. Herman and R.~S. Schumacher, ``Money Doesn't Grow on Trees, but Forecasts Do: Forecasting Extreme Precipitation with Random Forests,'' \emph{Monthly Weather Review}, vol.~146, no.~5, pp.~1571--1600, 2018.

\bibitem{b2}
H.~Sun \emph{et al.}, ``Corrected ERA5 precipitation by machine learning significantly improved flow simulations,'' \emph{Journal of Hydrometeorology}, vol.~23, no.~10, pp.~1663--1679, 2022.

\bibitem{b3}
B.~Yang \emph{et al.}, ``A method for monthly extreme precipitation forecasting,'' \emph{Water}, vol.~15, no.~8, 2023.

\bibitem{b4}
S.~Chkeir \emph{et al.}, ``Nowcasting extreme rain with machine learning techniques,'' \emph{Atmospheric Research}, vol.~282, 2023.

\bibitem{b5}
S.~Ravuri \emph{et al.}, ``Skillful precipitation nowcasting using deep generative models,'' \emph{Nature}, vol.~597, pp.~672--677, 2021.

\bibitem{b6}
G.~Ayzel \emph{et al.}, ``RainNet: A CNN for radar-based precipitation nowcasting,'' \emph{Geoscientific Model Development}, vol.~13, pp.~2631--2644, 2020.

\bibitem{b7}
F.~Gamboa-Villafruela, ``Immediate precipitation forecasting using deep learning,'' \emph{Proceedings}, vol.~8, no.~1, 2021.

\bibitem{b8}
D.~Wolfensberger and A.~Feinberg, ``RainForest: A random forest algorithm for precipitation estimation,'' \emph{Atmospheric Measurement Techniques}, vol.~14, pp.~3169--3193, 2021.

\bibitem{b9}
X.~Shi \emph{et al.}, ``Convolutional LSTM network for precipitation nowcasting,'' \emph{NeurIPS}, 2015.

\bibitem{b10}
X.~Shi, ``Smart dynamical downscaling of extreme precipitation,'' \emph{Geophysical Research Letters}, vol.~47, 2020.

\end{thebibliography}

\end{document}
